\documentclass{article}
\usepackage{amsmath}
\usepackage{amssymb}
\usepackage{gensymb}
\usepackage{mathrsfs}
\usepackage[utf8]{inputenc}
\usepackage[english]{babel}
\usepackage{amsthm}
\usepackage{multicol}

\renewcommand{\vec}[1]{\textbf{#1}}

\newtheorem{theorem}{Theorem}[section]
\newtheorem{corollary}{Corollary}[theorem]
\newtheorem{lemma}[theorem]{Lemma}

\theoremstyle{definition}
\newtheorem{definition}{Definition}[section]

\theoremstyle{remark}
\newtheorem*{remark}{Remark}

\theoremstyle{remark}
\newtheorem*{recall}{Recall}

\theoremstyle{remark}
\newtheorem*{it follows}{It follows}

\title{Multivariable Calculus}
\author{Andres Duarte}
\date{2018}

\begin{document}

\maketitle

\tableofcontents

\newpage

\section{Linear Equations in Linear Algebra}
\subsection{Systems of Linear Equations}

\begin{definition}
  A \textit{linear equation is an equation of the form}
  $$a_1 x_1 + a_2 x_2 + \dots + a_n x_n = b$$
\end{definition}

\begin{definition}
  A \textit{linear system} is a set of linear equations involving like variables.
\end{definition}

\begin{definition}
  A \textit{solution} to a linear system is an ordered set that makes the linear system true.
\end{definition}

\begin{definition}
  A \textit{solution set} is the set of all possible solutions to the linear system.
\end{definition}

\begin{remark}
  Two linear systems with like solution sets are \textit{equivalent}.
\end{remark}

\begin{remark}
  A linear system is \textit{consistent} if it has at least one solution, and \textit{inconsistent} if it has no solutions.
\end{remark}

\begin{definition}
  A \textit{coefficient matrix} is a matrix that consists of the coefficients of the variables of a linear system.
\end{definition}

\begin{remark}
  Each column of the coefficient matrix corresponds to a variable in the linear system.
\end{remark}

\begin{definition}
  An \textit{augmented matrix} consists of the coefficient matrix with an added column containing the constants of the RHS of the linear system.
\end{definition}

\begin{definition}
  An $m \times n$ \textit{matrix} is a rectangular array of elements with $m$ rows and $n$ columns.
\end{definition}

\subsubsection{Elementary Row Operations}

\begin{itemize}
  \item add the multiple of one row to another
  \item switch two rows
  \item scale a row by a nonzero constant
\end{itemize}

\begin{remark}
  Row operations are reversible.
\end{remark}

\begin{definition}
  Two matrices are \textit{row equivalent} if a sequence of row operations can transform one into the other.
\end{definition}

\begin{remark}
  All row equivalent augmented matrices have the same solution set.
\end{remark}

\subsubsection{Questions}

\begin{itemize}
  \item does a solution to the linear system exist?
  \item If it does, is it unique?
\end{itemize}

\subsection{Row Reduction and Echelon Forms}

\begin{definition}
  The \textit{leading entry} of a row is its left-most non-zero entry.
\end{definition}

\begin{definition}
  A matrix is in \textit{echelon form} if:
  \begin{itemize}
    \item all non-zero rows are above any all-zero rows
    \item the leading entry of each row is in a column to the right of the leading entry of the row above it
    \item all entries in a column below a leading entry are zeros
  \end{itemize}
\end{definition}

\begin{definition}
  A matrix is in \textit{reduced row echelon form} if:
  \begin{itemize}
    \item it's in echelon form
    \item all leading entries are 1
    \item all leading entries are the only non-zero entries in their columns
  \end{itemize}
\end{definition}

\begin{remark}
  A matrix can be row equivalent with many echelon forms but only one reduced echelon form.
\end{remark}

\begin{definition}
  A \textit{pivot position} corresponds to the position of one of the leading entries of the reduced echelon form of a matrix.
\end{definition}

\begin{definition}
  A column of the coefficient matrix is a \textit{free column} if it doesn't contain a pivot position.
\end{definition}

\begin{definition}
  A column of the augmented matrix is a \textit{pivot column} if it contains a pivot position.
\end{definition}

\begin{remark}
  Variables corresponding to free columns are \textit{free variables}. Variables corresponding to pivot columns are \textit{basic variables}.
\end{remark}

\begin{remark}
  The solution set of a consistent linear system has a \textit{parametric representation} in which by convention free variables act as parameters. The solution set of an incosistent linear system is empty and has \textbf{no} parametric representation.
\end{remark}

\begin{remark}
  Solving a system amounts to finding a parametric representation of the solution set or determing that the solution set is empty.
\end{remark}

\begin{remark}
  A linear system is consistent iff the right-most column of the augmented matrix is \textbf{not} a pivot column.
\end{remark}

\begin{remark}
  A consistent linear system has either a unique solution, if it has no free variables, or infinitely many solutions if it has at least one free variable.
\end{remark}

\subsection{Vector Equations}





\end{document}
