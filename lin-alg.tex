\documentclass{article}
\usepackage{amsmath}
\usepackage{amssymb}
\usepackage{textcomp}
\usepackage{gensymb}
\usepackage{mathrsfs}
\usepackage[utf8]{inputenc}
\usepackage[english]{babel}
\usepackage{amsthm}
\usepackage{multicol}

\renewcommand{\vec}[1]{\textbf{#1}}

\newtheorem{theorem}{Theorem}[section]
\newtheorem{corollary}{Corollary}[theorem]
\newtheorem{lemma}[theorem]{Lemma}

\theoremstyle{definition}
\newtheorem{definition}{Definition}[section]

\theoremstyle{remark}
\newtheorem*{remark}{Remark}

\theoremstyle{remark}
\newtheorem*{recall}{Recall}

\theoremstyle{remark}
\newtheorem*{it follows}{It follows}

\title{Multivariable Calculus}
\author{Andres Duarte}
\date{2018}

\begin{document}

\maketitle

\tableofcontents

\newpage

\section{Linear Equations in Linear Algebra}
\subsection{Systems of Linear Equations}

\begin{definition}
  A \textit{linear equation is an equation of the form}
  $$a_1 x_1 + a_2 x_2 + \dots + a_n x_n = b$$
\end{definition}

\begin{definition}
  A \textit{linear system} is a set of linear equations involving like variables.
\end{definition}

\begin{definition}
  A \textit{solution} to a linear system is an ordered set that makes the linear system true.
\end{definition}

\begin{definition}
  A \textit{solution set} is the set of all possible solutions to the linear system.
\end{definition}

\begin{remark}
  Two linear systems with like solution sets are \textit{equivalent}.
\end{remark}

\begin{remark}
  A linear system is \textit{consistent} if it has at least one solution, and \textit{inconsistent} if it has no solutions.
\end{remark}

\begin{definition}
  A \textit{coefficient matrix} is a matrix that consists of the coefficients of the variables of a linear system.
\end{definition}

\begin{remark}
  Each column of the coefficient matrix corresponds to a variable in the linear system.
\end{remark}

\begin{definition}
  An \textit{augmented matrix} consists of the coefficient matrix with an added column containing the constants of the RHS of the linear system.
\end{definition}

\begin{definition}
  An $m \times n$ \textit{matrix} is a rectangular array of elements with $m$ rows and $n$ columns.
\end{definition}

\subsubsection{Elementary Row Operations}

\begin{itemize}
  \item add the multiple of one row to another
  \item switch two rows
  \item scale a row by a nonzero constant
\end{itemize}

\begin{remark}
  Row operations are reversible.
\end{remark}

\begin{definition}
  Two matrices are \textit{row equivalent} if a sequence of row operations can transform one into the other.
\end{definition}

\begin{remark}
  All row equivalent augmented matrices have the same solution set.
\end{remark}

\subsubsection{Questions}

\begin{itemize}
  \item does a solution to the linear system exist?
  \item If it does, is it unique?
\end{itemize}

\subsection{Row Reduction and Echelon Forms}

\begin{definition}
  The \textit{leading entry} of a row is its left-most non-zero entry.
\end{definition}

\begin{definition}
  A matrix is in \textit{echelon form} if:
  \begin{itemize}
    \item all non-zero rows are above any all-zero rows
    \item the leading entry of each row is in a column to the right of the leading entry of the row above it
    \item all entries in a column below a leading entry are zeros
  \end{itemize}
\end{definition}

\begin{definition}
  A matrix is in \textit{reduced row echelon form} if:
  \begin{itemize}
    \item it's in echelon form
    \item all leading entries are 1
    \item all leading entries are the only non-zero entries in their columns
  \end{itemize}
\end{definition}

\begin{remark}
  A matrix can be row equivalent with many echelon forms but only one reduced echelon form.
\end{remark}

\begin{definition}
  A \textit{pivot position} corresponds to the position of one of the leading entries of the reduced echelon form of a matrix.
\end{definition}

\begin{definition}
  A column of the coefficient matrix is a \textit{free column} if it doesn't contain a pivot position.
\end{definition}

\begin{definition}
  A column of the augmented matrix is a \textit{pivot column} if it contains a pivot position.
\end{definition}

\begin{remark}
  Variables corresponding to free columns are \textit{free variables}. Variables corresponding to pivot columns of the coefficient matrix are \textit{basic variables}.
\end{remark}

\begin{remark}
  The solution set of a consistent linear system has a \textit{parametric representation} in which by convention free variables act as parameters. The solution set of an incosistent linear system is empty and has \textbf{no} parametric representation.
\end{remark}

\begin{remark}
  Solving a system amounts to finding a parametric representation of the solution set or determing that the solution set is empty.
\end{remark}

\begin{remark}
  A linear system is consistent iff the right-most column of the augmented matrix is \textbf{not} a pivot column.
\end{remark}

\begin{theorem}[Existence and uniqueness theorem]
  A linear system is consistent iff the right-most column of the augmented matrix is \textbf{not} a pivot column. \\
  A consistent linear system has either a unique solution, if it has no free variables, or infinitely many solutions if it has at least one free variable.
\end{theorem}

\subsection{Vector Equations}

\begin{align*}
  \mathbb{R}^n &:=\ \text{Set of ordered n-tuples}\
  \begin{bmatrix}
    x_1 \\
    \vdots  \\
    x_n
  \end{bmatrix}\
  \text{where}\ x_1, \dots, x_n \in \mathbb{R} \\
  \mathbb{R}^1 &:=\ \mathbb{R} = \text{Set of real numbers} =\text{Number line} \\
  \mathbb{R}^2 &:=\ \text{Plane} \\
  \mathbb{R}^3 &:=\ \text{Space}
\end{align*}

\begin{definition}
  A \textit{vector} in $\mathbb{R}^n$ is an element of $\mathbb{R}^n$.
\end{definition}

\begin{remark}
  Two vectors are equal iff their corresponding entries are equal.
\end{remark}

\subsubsection{Algebraic Properties of $\mathbb{R}^n$}

\begin{equation}
  \vec{u}, \vec{v}, \vec{w} \in \mathbb{R}^n\ \text{and}\ c, d \in \mathbb{R}
\end{equation}

\begin{itemize}
  \item $\vec{u} + \vec{v} = \vec{v} + \vec{u}$
  \item $(\vec{u} + \vec{v}) + \vec{w} = \vec{u} + (\vec{v} + \vec{w})$
  \item $\vec{u} + 0 = 0 + \vec{u} = \vec{u}$
  \item $\vec{u} + (-\vec{u}) = -\vec{u} + \vec{u} = 0$
  \item $c(\vec{u} + \vec{v}) = c\vec{u} + c\vec{v}$
  \item $(c + d)\vec{u} = c\vec{u} + d\vec{u}$
  \item $c(d\vec{u}) = (cd)\vec{u}$
  \item $1\vec{u} = \vec{u}$
\end{itemize}

\begin{definition}
  If $\vec{v}_1, \cdots, \vec{v}_p \in \mathbb{R}^n\ \text{and}\ c_1, \cdots, c_p \in \mathbb{R}$, then
  $$\vec{y} = c_1 \vec{v}_1 + \cdots + c_p \vec{v}_p$$
  is a linear combination of $\vec{v}_1, \cdots, \vec{v}_p$.
\end{definition}

\begin{definition}
  $\text{Span}\{\vec{v}_1, \cdots, \vec{v}_p\}$ is the set of all linear combinations of $\vec{v}_1, \cdots, \vec{v}_p$.
\end{definition}

\subsection{The Matrix Equation $Ax = b$}

\begin{definition}
  If $A^{m \times n}$ and $\vec{x} \in \mathbb{R}^n$ then $A\vec{x}=\vec{b}$ is the linear combination of the columns of $A$ using the corresponding entries of x as weights

  \begin{equation*}
    A\vec{x} = \left[ \, \vec{a}_1\ \cdots\ \vec{a}_n  \, \right]
    \begin{bmatrix}
      x_1 \\
      \vdots  \\
      x_n
    \end{bmatrix}\
    = x_1 \vec{a}_1 + \cdots + x_n \vec{a}_n
  \end{equation*}
\end{definition}

\begin{remark}
  $A\vec{x}$ is defined iff the number of columns of $A$ equals the number of entries in $\vec{x}$.
\end{remark}

\begin{theorem}
  If $A^{m \times n}$ and $\vec{b} \in \mathbb{R}^m$, then the \textit{matrix equation}
  $$A\vec{x} =\vec{b}$$
  has the same solution set as the \textit{vector equation}
  $$x_1\vec{a}_1 + \cdots + x_n\vec{a}_n = b$$
  which in turn, has the same solution set as the system of linear equations whose augmented matrix is
  $$ \left[ \, \vec{a}_1\ \cdots\ \vec{a}_n\ \vec{b}  \, \right]$$
\end{theorem}

\begin{remark}
  The equation $A\vec{x} = \vec{b}$ has a solution iff $\vec{b}$ is a linear combination of the columns of $A$.
\end{remark}

\begin{theorem}
  If $A^{m \times n}$, then the following statements are logically equivalent.
  \begin{itemize}
    \item $A\vec{x}=\vec{b}$ has a solution $\forall \vec{b} \in \mathbb{R}^m$
    \item Each $\vec{b} \in \mathbb{R}^m$ is a linear combination of the columns of $A$
    \item Span $\{\vec{a}_1, \cdots, \vec{a}_n \} = \mathbb{R}^m$
    \item $A$ has a pivot position in every row
  \end{itemize}
\end{theorem}

\begin{remark}
  The \textit{i}th entry of $A\vec{x}$ is the sum of the products of the entries of the \textit{i}th row of $A$ with the corresponding entries of $\vec{x}$
\end{remark}

\begin{definition}
  The \textit{identity matrix}, denoted $I_{n}$, is an $n \times n$ matrix with 1's on the diagonal and 0's elsewhere.
  $$I_{n} \, \vec{x} = \vec{x}\ \forall \vec{x} \in \mathbb{R}^n$$
\end{definition}

\begin{theorem}
  If $A^{m \times n}$, $\vec{u}\ \text{and}\ \vec{v} \in \mathbb{R}^n$ and c \in \mathbb{R}, then
  \begin{itemize}
    \item $A(\vec{u} + \vec{v}) = A\vec{u} + A\vec{v}$
    \item $A(c\vec{u}) = c(A\vec{u})$
  \end{itemize}
\end{theorem}

\subsection{Solution Sets of Linear Systems}

\begin{definition}
  A system of linear equations is \textit{homogeneous} if it can be written in the form
  $$A \vec{x} = \vec{0}$$
  where $A^{m \times n}$ and $\vec{0} \in \mathbb{R}^m$
\end{definition}

\begin{remark}
  All homogeneous linear systems, $A\vec{x} = \vec{0}$, have a trivial solution.
\end{remark}

\begin{it follows}
  From the existence and uniqueness theorem. \\
  A homogeneous linear system has a non-trivial solution iff it has at least one free variable.
\end{it follows}

\begin{definition}
  The parametric vector form of the solution set of the homogeneous linear system $A\vec{x} = \vec{0}$ with one free variable is
  $$ \vec{x} = t\vec{v} \qquad \text{where}\ t \in \mathbb{R}$$
\end{definition}

\begin{remark}
  The solution set of a homogeneous linear system, $A\vec{x} = \vec{0}$, with only one free variable is a line through the origin.
\end{remark}

\begin{definition}
  The parametric vector form of the solution set of the non-homogeneous linear system $A\vec{x} = \vec{b}$ is
  $$ \vec{x} = \vec{p} + t\vec{v} \qquad \text{where}\ t \in \mathbb{R}$$
\end{definition}

\begin{theorem}
  The solution set of a consistent non-homogeneous linear system $A\vec{x} = \vec{b}$ is obtained by translating the solution set of its corresponding homogeneous linear system $A\vec{x} = \vec{0}$ by any particular solution $\vec{p}$ of $A\vec{x} = \vec{b}$.
\end{theorem}

\begin{definition}
  An indexed set of vectors $\{\vec{v}_1, \cdots, \vec{v}_p \} \in \mathbb{R}^n$ is \textit{linearly independent} if the vector equation
  $$x_1\vec{v}_1 + \cdots + x_p\vec{v}_p = \vec{0}$$
  has only the trivial solution.
\end{definition}

\begin{definition}
  An indexed set of vectors $\{\vec{v}_1, \cdots, \vec{v}_p \} \in \mathbb{R}^n$ is \textit{linearly dependent} if $\exists$ weights $c_1, \cdots, c_p$, not all zero, such that
  $$c_1 \vec{v}_1 + \cdots + c_p \vec{v}_p = \vec{0}$$
\end{definition}

\begin{remark}
  The columns of $A$ are linearly independent iff the equation $A\vec{x} = \vec{0}$ has only the trivial solution.
\end{remark}

\begin{remark}
  An indexed set of vectors is linearly dependent iff at least one of the vectors in the set is a linear combination of the others.
\end{remark}

\begin{remark}
  A set containing a single vector is linearly independent iff its not the zero vector. It follows: the zero vector is linearly dependent.
\end{remark}

\begin{theorem}
  Any set $\{\vec{v}_1, \cdots, \vec{v}_p \} \in \mathbb{R}^n$ is linearly dependent if $p > n$
\end{theorem}

\begin{theorem}
  Any set $\{\vec{v}_1, \cdots, \vec{v}_p \} \in \mathbb{R}^n$ that contains the zero vector is linearly dependent.
\end{theorem}

\subsection{Introduction to Linear Transformations}

\begin{definition}
  A \textit{transformation} $T$ from $\mathbb{R}^n$ to $\mathbb{R}^m$, written $T: \mathbb{R}^n \to \mathbb{R}^m$, is a rule that assigns to each vector $\vec{x} \in \mathbb{R}^n$ a vector $T(\vec{x}) \in \mathbb{R}^m$.
  \begin{itemize}
    \item $\mathbb{R}^n$ is the \textit{domain} of $T$ and $\mathbb{R}^m$ is its \textit{codomain}.
    \item $\forall\ \vec{x} \in \mathbb{R}^n$ the vector $T(\vec{x}) \in \mathbb{R}^m$ is the \textit{image} of x.
    \item The set of all  images $T(\vec{x})$ is the \textit{range} of $T$.
  \end{itemize}
\end{definition}

\begin{remark}
  The codomain is all the places the transformation could take you, the range is all the places it does.
\end{remark}

\begin{remark}
  $\forall\ \vec{x} \in \mathbb{R}^n$, $T(\vec{x})$ is computed as $A\vec{x}$ and written  as $\vec{x} \mapsto A\vec{x}$.
\end{remark}

\begin{remark}
  The range of $T$ is the set of all linear combinations of the columns of $A$.
\end{remark}

\begin{definition}
  A transformation $T$ is \textit{linear} if:
  \begin{itemize}
    \item $T(\vec{u} + \vec{v}) = T(\vec{u}) + T(\vec{v}) \qquad \forall\ \vec{u}, \vec{v}$ in the domain of $T$.
    \item $T(c\vec{u}) = c \, T(\vec{u}) \qquad \forall\ c, \vec{u}$ in the domain of $T$.
  \end{itemize}
\end{definition}

\begin{remark}
  Every matrix transformation is a linear transformation.
\end{remark}

\begin{remark}
  Linear Transformations preserve the operations of vector addition and scalar multiplication.
\end{remark}

\begin{it follows}
  From linearity.
  $$T(\vec{0}) = 0$$
\end{it follows}


\end{document}
